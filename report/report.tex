\documentclass[11pt]{article}
\input{\string~/.macros}
\usepackage[a4paper, total={6in, 8in}, margin=1.5in]{geometry}
\usepackage{hyperref}
\hypersetup{colorlinks=true, linktoc=all, linkcolor=blue}

% \linespread{1.5}
 
\newcommand{\heading}[1]{(#1)}
\newcommand{\bheading}[1]{\textbf{(#1)}}

\newcommand{\bsigma}{\boldsymbol{\sigma}}
\newcommand{\bepsilon}{\boldsymbol{\varepsilon}}
\renewcommand{\epsilon}{\varepsilon}
\renewcommand{\bf}{\mathbf{f}}
\newcommand{\ta}{\tilde{a}}
\renewcommand{\bphi}{\boldsymbol{\varphi}}

\begin{document}

\section*{Abstract}

We aim to determine failure cases, i.e. fractures, breakage, etc. for solid statues of arbitrary elastic material in real time. To solve this problem, we aim to perform linear elasticity simulation and determine yield stress for a solid geometry. To make the solution fast enough, we aim to use a combination of voxel grids and matrix-free iterative method to allow for parallelism. Briefly, we will
\begin{enumerate}
    \item provide background introduction and model assumptions
    \item describe differential equations governing linear elasticity
    \item discuss strong and weak solutions and clarify implementation strategies
    \item explore tricks to make simulation faster
\end{enumerate}


\section*{Introduction \& Definitions}

Continuum mechanics deals with behavior of material modeled as a continuous mass. The study of elastic material is a subset of continuum mechanics that models the ability of a material to resist influence and return to its original configuration when the influence is removed. The following definitions are important to elasticity theory and provide important background for rest of the report.

\begin{enumerate}
    \item \bheading{stress} describes internal force over a differential area at a point, which can be represented as a symmetric second order tensor $\bsigma$,
    \[
        \bsigma = 
        \begin{bmatrix}
            \sigma_{11} & \sigma_{12} & \sigma_{13} \\
            \sigma_{21} & \sigma_{22} & \sigma_{23} \\
            \sigma_{31} & \sigma_{32} & \sigma_{33} \\
        \end{bmatrix}    
    \]
    where
    \[
        \sigma_{ij} = \lim_{dA_i \to 0} \dfrac{F_j}{dA_i}
    \] 
    represents force applied in $j$-th coordinate axis over a differential plane orthogonal to $i$-th coordinate axis. 
    \item \bheading{strain} describes deformation by relative displacements, which can be represented as a symmetric second order tensor $\bepsilon$,
    \[
        \bepsilon = 
        \begin{bmatrix}
            \epsilon_{11} & \epsilon_{12} & \epsilon_{13} \\
            \epsilon_{21} & \epsilon_{22} & \epsilon_{23} \\
            \epsilon_{31} & \epsilon_{32} & \epsilon_{33} \\
        \end{bmatrix}
    \]
    where $\epsilon_{ii}$ are axial strains and $\epsilon_{ij}$ where $i\neq j$ are shearing strains.
    \item \bheading{displacement} describes how much a point moves from referenced to deformed configuration, which can be represented as component-wise displacement, $\bu$
    \[
        \bu = \p{u_1, u_2, u_3}
    \]
    \item \bheading{yield} a point on a stress-strain curve of a material that indicates the limit of elastic behavior. Material experiencing stress over the yield point will suffer from permanent deformations.
\end{enumerate}


\section*{Goal}

Ultimately, we are interested in computing the stress field over all points within a solid geometry, from which we can deduce material yielding by comparing the effective stress $\sigma_e$ with material-specific yield stress $\sigma_y$. Simply, we use Von Mises' criterion to determine yielding, 
\[
    \sigma_e = \frac{1}{\sqrt{2}}\sqrt{
        (\sigma_{11} - \sigma_{22})^2 
        + (\sigma_{22} - \sigma_{33})^2 
        + (\sigma_{33} - \sigma_{11})^2
        + 6(\sigma_{23}^2 + \sigma_{31}^2 + \sigma_{12}^2)
    } > \sigma_y
\]


\section*{Problem Assumptions}

Here, we lay down simplifying assumptions for our problem.
\begin{enumerate}
    \item \bheading{Analytic stress/strain} If we assume that stress and strain field is analytic in small neighborhood of a point, we can guarantee convergent Taylor series expansion, which leads to simplier formulation of differential equation governing materials' elastic behavior. This assumption, for example, helped with derivation of Green-Lagrangian strain tensor
    \[
        \epsilon_{ij} = \p{u_{i,j} + u_{j,i} - u_{k,i}u_{k,j}}    
    \]
    \item \bheading{Small deformations} Typically, elastic material exhibit a nonlinear relationship between stress and strain. However, if we assume for small deformation only, stress-strain is linear, hence the name \textit{linear} elasticity. We can employ Hooke's law to model the relationship between stress and strain.
    \[
        \bsigma = \bC : \bepsilon
    \]
    where $\bC$ is a fourth-order stiffness tensor.
    \item \bheading{Isotropic homogeneous material} Typically, material exhibit direction or location specific behavior. If materials are characterized by properties independent of orientation and coordinate system, we can reduce the number of material-dependent constants to 2 (i.e. $\lambda,\mu$),
    \[
        \bsigma = \bC : \bepsilon
        \quad \quad \rightarrow \quad \quad
        \bsigma = 2\mu\bepsilon + \lambda \tr{\bepsilon} \bI
    \]
    \item \bheading{Static equilibrium} We are only interested in modeling static objects. With this assumption, we obtain a simpler formula from Newton's second law on conservation of linear and angular momentum.
    \[
        \nabla\cdot\bsigma + \bf = \rho \ddot{\bu}
        \quad \quad \rightarrow \quad \quad
        \nabla\cdot\bsigma + \bf = 0
    \]
    where $\bf$ are body forces, and $\rho$ is density of the material.
\end{enumerate}


\section*{Differential Equations}

From physical laws and simplifying assumptions, we can derive a set of equations governing linear elasticity with respect to the stress field $\bsigma$, the strain field $\bepsilon$, and the displacement field $\bu$. We express the equations in tensor notations.
\begin{enumerate}
    \item \bheading{Equilibrium Equations} express equilibirum condition imposed to the stress field by Newton's second law
    \[
        \nabla\cdot\bsigma + \bf = 0
        \quad\quad\quad\quad
        \sigma_{ji,j} + f_i = 0
    \]
    \item \bheading{Kinematics Equations} express deformation with respect to displacement without reference to forces that created them. 
    \[
        \bepsilon = \dfrac{1}{2} \p{
            \nabla \bu + (\nabla\bu)^T
        }
        \quad\quad\quad\quad
        \epsilon_{ij} = 
            \frac{1}{2} \p{
                \frac{\partial u_i}{\partial x_j} + \frac{\partial u_j}{\partial x_i}}
    \]
    \item \bheading{Constitutive Law} describes behavior of material under stress. We approximate behavior of linear elastic material with Hooke's Law
    \[
        \bsigma = \bC : \bepsilon
        \quad\quad\quad\quad
        \sigma_{ij} = C_{ijkl} \epsilon_{kl}
    \]
    where $\bC$ is a 4-th order stiffness tensor encoding material properties. Specifically, under the assumption of isotropic material, the relationship is simplified to 
    \[
        \bsigma = 2\mu\bepsilon + \lambda \tr{\bepsilon} \bI
        \quad\quad\quad\quad
        \sigma_{ij} = 2\mu \epsilon_{ij} + \lambda \epsilon_{kk} \delta_{ij}
    \]
    where $\lambda,\mu$ are Lam\'{e}'s first and second parameter can be written as material-specific constants - Poisson's ratio $\nu$ and Young's modulus $E$,
    \[
        \mu = \frac{E}{2(1+\nu)}
        \quad\quad\quad\quad
        \lambda = \frac{E\nu}{(1+\nu)(1-2\nu)}
    \]
\end{enumerate}


\section*{Boundary Value Problem}
For our purposes, it is easier to eliminate stress and strain to seek a set of equations where the displacement field is the primary unknown. We can achieve this by writing the strain $\bsigma$ in the equilibrium equation as a function of displacement $\bu$ using the constitutive law and kinematics equations. In other words, we seek displacement field $\bu \in \mathring{\BH}$ satisfying
\begin{align*}
    -\nabla \cdot \p{\bC \bepsilon(\bu)}
        &= \bf & \bx\in\Omega \\
    \bu &= 0   &\bx\in\partial\Omega
\end{align*}
where $\Omega\in\R^3$ is a closed and bounded region occupied by a linear elastic solid, with boundary $\partial\Omega$. $\mathring{\BH} := \mathring{\BH}(\Omega,\R^3)$ is the first Sobolev space satisfying the zero boundary condition. $\bepsilon$ is a linear differential operator over the displacement $\bu$, while $\bC$ is a linear operator over the the strain $\bepsilon(\bu)$, which is positive definite for linear elastic isotropic material. It is easy to see that $\sL := -\nabla \cdot \p{\bC \bepsilon(\cdot)}$ is a linear differential operator. So we have,
\begin{align*}
    \sL \bu &= \bf  & \bx\in\Omega \\
    \bu   &= 0    &\bx\in\partial\Omega
\end{align*}


\section*{Finite Element Formulation}
It is conventional to use finite element method to solve linear elasticity equations for 3-dimensional objects with complex geometry. We will formulate the boundary value problem in both the weak and variational form. The two formulations are equivalent in our setup.

\subsection*{Weak Form}
The goal is to estimate $\bu$ in a finite dimensional function space $\mathring{\BH}_m$ spanned by a set of basis functions $\pc{\bv_1, \cdots, \bv_m} \subset \mathring{\BH}$. We can formulate linear elasticity weakly as
\[
    \int_{\Omega} \bC\bepsilon(\bu) : \bepsilon(\bv_k) d\bx
    = 
    \int_{\Omega} \bf \cdot \bv_k d\bx
    \tag{$k=1,\cdots,m$}
\]
where the left hand side is obtained via Green's first identity assuming a pure displacement formulation with no traction force on the boundary. We can define a bilinear form $\ta(\cdot,\cdot)$
\[
    \ta(\bu,\bv) = \inner{\sL \bu}{\bv}
        = \int_{\Omega} \bC \bepsilon(\bu) : \bepsilon(\bu) d\bx    
\]
where $\inner{\cdot}{\cdot}$ is $L^2$ norm over $\Omega$. We can rewrite the Galerkin's Equations as
\[
    \ta(\bu,\bv_k) = \inner{\bf}{\bv_k}
    \tag{$k=1,\cdots,m$}
\]
\subsection*{Variational Form}
We can show that $\sL$ is positive definite [insert proof here]. By theorem [cite textbook], $\sL \bu = \bf$ is the Euler-Lagrange equation of following variational problem
\[
    \sJ(\bv) := \ta(\bv,\bv) - 2\inner{\bf}{\bv}
    \quad\quad\quad
    \bv\in\BH
\]
where $\sJ: \BH\to \R$ is a functional. The theorem also guarantees the uniqueness and existence of solution for the boundary value problem. Therefore, we want to solve the following optimization problem
\[
    \min_{\bu\in \BH} \quad
        \int_{\Omega} \bC \bepsilon(\bu) : \bepsilon(\bu) d\bx 
      - 2\int_{\Omega} \bf \cdot\bu d\bx
\]





\end{document}
